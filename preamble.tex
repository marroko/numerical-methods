\documentclass[a4paper,onecolumn,10pt]{article}
\usepackage[polish]{babel}
\usepackage[utf8]{inputenc}
\usepackage[T1]{fontenc}
\usepackage[left=2.1cm,right=2.1cm]{geometry}
\usepackage[dvipsnames]{xcolor}
\usepackage{amsmath,calc,indentfirst,fancyhdr,amsfonts,graphicx,epstopdf,caption, mathcomp, subcaption,wrapfig, siunitx,pbox,float,algorithm}
\usepackage[noend]{algpseudocode}


\makeatletter
\def\BState{\State\hskip-\ALG@thistlm}
\renewcommand{\ALG@name}{Algorytm}
\makeatother

\renewcommand{\baselinestretch}{1.1}	 % odstep miedzy liniami
\addto\captionspolish{\renewcommand{\figurename}{Wykres}} % zmiana podpisu pod obrazkami, zamiast "Rysunek" bedzie "Wykres"
\newcommand{\NN}{\mathbb{N}}			 % makro do znaku liczb naturalnych

\newcommand{\R}[1]{\textcolor{red}{#1}}  % makro do polecenia z parametrami - tutaj 1 parametr
\newcommand{\G}[1]{\textcolor{green}{#1}} 
\newcommand{\B}[1]{\textcolor{RoyalBlue}{#1}} 
% kolorowanie {\B{argument}}

\newcommand{\PICTURES}{} % szybsza kompilacja dzieki stalej "usuwajacej" obrazki
						 % zakomentowanie \PICTURES powoduje znikniecie obrazkow

\pagestyle{fancy} % formatuj caly dokument
\fancyhead{}
\fancyfoot{}
\renewcommand{\headrulewidth}{0pt}
\fancyfoot[R]{\thepage} % dla stron poza tytulowa nr w prawym dolnym rogu

\fancypagestyle{plain}{ % dla strony tytulowej nr w prawym dolnym rogu
	
	\renewcommand{\headrulewidth}{0pt}
	\fancyhf{}
	\fancyfoot[R]{\thepage}
}