\documentclass[a4paper,onecolumn,10pt]{article}
\usepackage[polish]{babel}
\usepackage[utf8]{inputenc}
\usepackage[T1]{fontenc}
\usepackage[left=2.1cm,right=2.1cm]{geometry}
\usepackage[dvipsnames]{xcolor}
\usepackage{amsmath,calc,indentfirst,fancyhdr,amsfonts,graphicx,epstopdf,caption, mathcomp, subcaption,wrapfig, siunitx,pbox,float,algorithm}
\usepackage[noend]{algpseudocode}


\makeatletter
\def\BState{\State\hskip-\ALG@thistlm}
\renewcommand{\ALG@name}{Algorytm}
\makeatother

\renewcommand{\baselinestretch}{1.1}	 % odstep miedzy liniami
\addto\captionspolish{\renewcommand{\figurename}{Wykres}} % zmiana podpisu pod obrazkami, zamiast "Rysunek" bedzie "Wykres"
\newcommand{\NN}{\mathbb{N}}			 % makro do znaku liczb naturalnych

\newcommand{\R}[1]{\textcolor{red}{#1}}  % makro do polecenia z parametrami - tutaj 1 parametr
\newcommand{\G}[1]{\textcolor{green}{#1}} 
\newcommand{\B}[1]{\textcolor{RoyalBlue}{#1}} 
% kolorowanie {\B{argument}}

\newcommand{\PICTURES}{} % szybsza kompilacja dzieki stalej "usuwajacej" obrazki
						 % zakomentowanie \PICTURES powoduje znikniecie obrazkow

\pagestyle{fancy} % formatuj caly dokument
\fancyhead{}
\fancyfoot{}
\renewcommand{\headrulewidth}{0pt}
\fancyfoot[R]{\thepage} % dla stron poza tytulowa nr w prawym dolnym rogu

\fancypagestyle{plain}{ % dla strony tytulowej nr w prawym dolnym rogu
	
	\renewcommand{\headrulewidth}{0pt}
	\fancyhf{}
	\fancyfoot[R]{\thepage}
}

\title{\Large\vspace{-2.5cm}{\Huge S}PRAWOZDANIE - LABORATORIUM NR {\Huge2}\\
		\textbf{Odwracanie macierzy, obliczanie wyznacznika \\i wskaźnika uwarunkowania macierzy \\przy użyciu rozkładu LU } } 
\date{\Large7 marca 2019}
\author{\Large Marek Kiełtyka}

\begin{document}
\maketitle

\section{Wstęp}
	
\subsection{Rozkład LU}

Zasadniczym celem metody jest znalezienie takich dwóch macierzy trójkątnych \textbf{L}, \textbf{U} dla których prawdziwa jest równość
\begin{equation}
\boldsymbol{A = L \cdot U},
\label{lu}
\end{equation}
gdzie \textbf{A} - wyjściowa macierz. Z kolei pozostałe oznaczenia pochodzą od angielskich nazw:
\begin{itemize}
	\item \textit{Lower} - dolna z jedynkami na głównej przekątnej,
	\item \textit{Upper} - górna z niezerowymi elementami na głównej przekątnej.
\end{itemize}
 Metoda ta znacznie ułatwia rozwiązywanie układów równań liniowych i znajdowanie właściwości danej macierzy, jednak główny nacisk położony jest na uzyskanie równości (\ref{lu}). Najlepiej wykorzystać poniższą metodę.

\subsection{Wyznaczanie macierzy LU metodą Gaussa}

Analogicznie jak przy obliczaniu macierzy odwrotnej, należy dokonać zestawienia $ \left[\boldsymbol{I}\right]\left[\boldsymbol{A}\right] $. Co ważne, kolejność jest inna niż w tamtym przypadku. 
Ponadto nie chodzi tutaj o dążenie do $ \left[\boldsymbol{I}\right]\left[\boldsymbol{A^{-1}}\right]. $
Główny problem stanowi uzyskanie macierzy \textbf{L}, \textbf{U} w danym zestawieniu w odpowiednich miejscach przy użyciu dozwolonych tą metodą sposobów (dodawanie wierszy, mnożenie przez skalar etc.). Zatem po przeprowadzeniu toku rozumowania spodziewanym wynikiem będzie:

\begin{equation}
\begin{bmatrix}
1 		& 0 	  & 0 	    & \dots  & 0 	\\  
l_{2,1} & 1 	  & 0 	    & \dots  & 0	 \\
l_{3,1} & l_{3,2} & 1 	    & \dots  & 0 	  \\
\vdots  & \vdots  & \vdots  & \ddots & \vdots \\
l_{n,1} & l_{n,2} & l_{n,3} & \dots  & 1 	 
\end{bmatrix}
\begin{bmatrix}
u_{1,1} & u_{1,2} & u_{1,3} & \dots & u_{1,n} \\
0 & u_{2,2} & u_{2,3} & \dots & u_{2,n} \\
0 & 0 & u_{3, 3} & \dots & u_{3, n} \\
\vdots & \vdots & \vdots & \ddots & \vdots \\
0 & 0 & 0 & \dots & u_{n,n}
\end{bmatrix}
\end{equation}
Innymi słowy jest to równoważna postać prawej strony równania (\ref{lu}).

\subsection{Wyznacznik rozłożonej macierzy}

Z właśności wyznacznika:
\begin{equation}
det(A) = det(LU) = det(L) \cdot det(U) = det(U) \implies det(A) = \prod_{i=1}^{n} u_{i,i}
\label{wyznacznik}
\end{equation}
Przekształcenia w wyrażeniu (\ref{wyznacznik}) były możliwe dzięki obecności jedynek na diagonali macierzy \textbf{L}. Finalnie, wyznacznik macierzy wyjściowej można obliczyć jako iloczyn elementów głównej przekątnej macierzy \textbf{U}. 

\subsection{Rozwiązywanie URL}
Biorąc pod uwagę (\ref{lu}) można przedstawić układ równań liniowych jako

\begin{equation}
A\vec{x} =\vec{b}
\implies
LU\vec{x} =\vec{b}.
\label{pojedynczy}
\end{equation}
Zakładając $ U\vec{x} =\vec{y} $ wystarczy rozwiązać układ równań zawierający macierze trójkątne.
\begin{equation}
\begin{cases}
L\vec{y} =\vec{b} \\
U\vec{x} =\vec{y}
\end{cases}
\end{equation}
Obliczenie $ \vec{y} $ pozwala uzyskać rozwiązanie $ \vec{x} $ i czyni to szybciej niż w przypadku pojedynczego układu (\ref{pojedynczy}).

\subsection{Wskaźnik uwarunkowania macierzy}
Jest to wielkość mówiąca o tym, jak bardzo dane wejściowe mogą zaburzyć poszukiwane rozwiązanie (tj. wpłynąć na błędy wyniku). Najlepiej wyznaczyć ją korzystając z\,relacji:
\begin{equation}
\kappa(A)= \begin{Vmatrix} A \end{Vmatrix} \cdot \begin{Vmatrix} A^{-1} \end{Vmatrix}
\end{equation}
i używając dodatkowo normy zdefiniowanej jako największa bezwzględna wartość danej macierzy.
 
\section{Zadanie do wykonania}

\subsection{Opis problemu}

Na zajęciach laboratoryjnych dokonano analizy macierzy kwadratowej o rozmiarze $N = 4$, której elementy zdefiniowano jako: $a_{i,j} = \frac{1}{i + j + \delta}$ dla $i,j \in \{1, 2, 3, 4\}$. Przyjęto parametr $\delta = 0$, bo skorzystano z biblioteki \textit{Numerical Recipes}. Wzorem wstępu teoretycznego analiza składała się z wyznaczenia:
\begin{itemize}
	\item macierzy LU, a także każdej z osobna
	\item wyznacznika macierzy wyjściowej 
	\item macierzy odwrotnej
	\item iloczynu $ AA^{-1} $
	\item norm tych macierzy, a w konsekwencji wskaźnika uwarunkowania macierzy\,wyjściowej
\end{itemize}

\subsection{Wyniki}

Korzystając z wyżej wspomnianej biblioteki zawierającej m.in. poniższe procedury:
\begin{center}
	\textit{
void ludcmp(float **matrix, int size, int *indx, float *d); \\
void lubksb(float **matrix, int size, int *indx, float *vector); }
\end{center}
napisano program w języku C++ celem dokonania analizy. Pierwsza z funkcji nadpisuje wejściową macierz jej rozkładem LU, aby zaoszczędzić pamięć komputera. Jednak na potrzeby zadania wykonano kopię, aby móc sprawdzać zawartość obu macierzy równocześnie. 

Funkcja druga umożliwiła znalezienie macierzy odwrotnej dzięki rozwiązaniu czterech układów równań postaci $ A\vec{x} =\vec{b} $. Tam kolejno zmieniał się tylko wektor wyrazów wolnych w sposób następujący:

\begin{equation}
b_1 = 
\begin{bmatrix}
1\\0\\0\\0
\end{bmatrix}
b_2 = 
\begin{bmatrix}
0\\1\\0\\0
\end{bmatrix}
b_3 = 
\begin{bmatrix}
0\\0\\1\\0
\end{bmatrix}
b_4 = 
\begin{bmatrix}
0\\0\\0\\1
\end{bmatrix}
\end{equation}
Wektor $ \vec{x} $ zawierał w sobie daną kolumnę macierzy odwrotnej w kolejnych iteracjach. Z\,kolei do mnożenia macierzy wykorzystano klasyczny algorytm iterujący po kolumnach i\,wierszach.

Wszystkie wyniki przekierowano do pliku jako strumień wyjściowy wykonania programu i zaprezentowano poniżej.

\begin{equation*}
L = 
\begin{bmatrix}
1 &0& 0& 0 \\
2.5 &1 &0& 0 \\
1.(6)& 0.(3)& 1& 0 \\
1.25 &0.1 &0.499997 &1 
\end{bmatrix}
U = 
\begin{bmatrix}
0.2& 0.1(6) &0.142857& 0.125 \\
0 &-0.08(3)& -0.107143& -0.1125\\ 
0 &0 &-0.00238095& -0.00416666 \\
0 &0& 0& -5.95264\times 10^{-5}
\end{bmatrix}
\end{equation*}

\begin{center}
	$ det(A) = -2.36216 \times 10^{-9} $
\end{center}

\begin{equation*}
A^{-1} = 
\begin{bmatrix}
199.992 &-1199.95 &2099.9 &-1119.95 \\
-1199.94 &8099.62 &-15119.3& 8399.65 \\
2099.88 &-15119.3 &29398.7 &-16799.3 \\ 
-1119.93 &8399.58& -16799.3 &9799.61 
\end{bmatrix}
\end{equation*}

\begin{equation*}
AA^{-1} = 
\begin{bmatrix}
0.999985 &0 &-0.000976562 &0.000244141 \\
-1.52588\times10^{-5} &1 &-0.000732422& 0.000244141\\ 
0& 0.00012207 &0.999756 &0.000244141 \\
0 &0& 0& 1 
\end{bmatrix}
\end{equation*}

$$ \kappa(A)= \begin{Vmatrix} A \end{Vmatrix} \cdot \begin{Vmatrix} A^{-1} \end{Vmatrix} = 0.5\cdot29398.7 = 14699.4 $$

\section{Wnioski}

Metoda LU czyni dużo łatwiejszymi do obliczenia: wyznacznik, macierz odwrotną i\,zależny od niej wskaźnik uwarunkowania macierzy wyjściowej. Nie jest przy tym pozbawiona wad. Wystarczy spojrzeć na iloczyn $ AA^{-1} $, który powinien być dokładnie równy macierzy jednostkowej. Jednym z powodów może być bardzo duży wskaźnik uwarunkowania. Po części za otrzymany wynik odpowiada również komputerowa reprezentacja liczb zmiennoprzecinkowych.

Dodatkowo niski rząd wielkości elementów macierzy $ U $ na głównej przekątnej przyczynia się do wartości wyznacznika badanej macierzy oscylującej w granicach zera. 

Wszystko to świadczy o złym uwarunkowaniu zadania. Warto korzystać z metody LU w obliczeniach numerycznych, choć jak pokazało niniejsze laboratorium, nie zawsze daje to korzystne rezultaty.

\end{document}