\documentclass[a4paper,onecolumn,10pt]{article}
\usepackage[polish]{babel}
\usepackage[utf8]{inputenc}
\usepackage[T1]{fontenc}
\usepackage[left=1.2cm,right=1.2cm]{geometry}
\usepackage[dvipsnames]{xcolor}
\usepackage{amsmath,calc,indentfirst,fancyhdr,amsfonts,graphicx,epstopdf,caption, mathcomp, subcaption,wrapfig, siunitx,pbox,float,algorithm}
\usepackage[noend]{algpseudocode}


\makeatletter
\def\BState{\State\hskip-\ALG@thistlm}
\renewcommand{\ALG@name}{Algorytm}
\makeatother

\renewcommand{\baselinestretch}{1.1}	 % odstep miedzy liniami
\addto\captionspolish{\renewcommand{\figurename}{Wykres}} % zmiana podpisu pod obrazkami, zamiast "Rysunek" bedzie "Wykres"
\newcommand{\NN}{\mathbb{N}}			 % makro do znaku liczb naturalnych

\newcommand{\R}[1]{\textcolor{red}{#1}}  % makro do polecenia z parametrami - tutaj 1 parametr
\newcommand{\G}[1]{\textcolor{green}{#1}} 
\newcommand{\B}[1]{\textcolor{RoyalBlue}{#1}} 
% kolorowanie {\B{argument}}

\newcommand{\PICTURES}{} % szybsza kompilacja dzieki stalej "usuwajacej" obrazki
						 % zakomentowanie \PICTURES powoduje znikniecie obrazkow

\pagestyle{fancy} % formatuj caly dokument
\fancyhead{}
\fancyfoot{}
\renewcommand{\headrulewidth}{0pt}
\fancyfoot[R]{\thepage} % dla stron poza tytulowa nr w prawym dolnym rogu

\fancypagestyle{plain}{ % dla strony tytulowej nr w prawym dolnym rogu
	
	\renewcommand{\headrulewidth}{0pt}
	\fancyhf{}
	\fancyfoot[R]{\thepage}
}

\title{\Large\vspace{-2.5cm}{\Huge S}PRAWOZDANIE - LABORATORIUM NR {\Huge4}\\
		\textbf{Wyznaczanie wartości i wektorów własnych\\ macierzy symetrycznej} } 
\date{\Large21 marca 2019}
\author{\Large Marek Kiełtyka}

\begin{document}
\maketitle

\section{Wstęp}
	
\subsection{Redukcja Householdera}

Tytułowy problem własny można znacząco uprościć, redukując macierz symetryczną $ A $ do postaci trójdiagonalnej $ T $. Zależy nam na takim rezultacie:
\begin{equation}
\begin{pmatrix}
a_1 & b_1 & c_1 & d_1 & e_1 \\ 
b_1 & a_2 & b_2 & \ddots & d_2 \\
c_1 & b_2 & a_3 & \ddots & c_{n-2} \\
d_1 & \ddots & \ddots & \ddots& b_{n-1}\\
e_1 & d_2 & c_{n-2} & b_{n-1} & a_n
\end{pmatrix}
\implies
\begin{pmatrix}
f_1 & g_1 & 0 & 0 & 0 \\ 
g_1 &f_2 &g_2 & \ddots & 0 \\
0 & g_2 & f_3 & \ddots & 0 \\
0 & \ddots & \ddots & \ddots& g_{n-1}\\
0 & 0 &0 & g_{n-1} &f_n
\end{pmatrix}
\label{przejscie}
\end{equation}
Pośrednim krokiem jest przekształcenie macierzy wejściowej $ A $ do macierzy $ P $, które razem dają następujący iloczyn:

\begin{equation}
T = P^{-1}AP
\label{tpap}
\end{equation}
Metoda jest stabilna numerycznie, więc nawet w przypadku ewentualnej kumulacji błędów nie dojdzie do zakłamania wyniku.

\subsection{Wartości i wektory własne macierzy trójdiagonalnej}

Z definicji wektor $ \vec{y} $ jest wektorem własnym macierzy T jeśli istnieje taka liczba $\lambda$ będąca wartością własną macierzy T, dla której 
\begin{equation}
 T\vec{y} = \lambda \vec{y}.
 \label{ty}
\end{equation}
Dzięki przekształceniu (\ref{przejscie}) uzyskuje się przy okazji wartości własne dla macierzy $ A $, które są równe obliczonym kolejno $\lambda_k$. Ponadto wyznaczenie wartości i wektorów własnych jest dużo szybsze z uwagi na postać macierzy $ T $.

\subsection{Wektory własne macierzy symetrycznej}

Korzystając z równań (\ref{tpap}) oraz (\ref{ty}) dokonuje się równoważnych przekształceń:
\begin{alignat*}{4}
P^{-1}AP\vec{y} &= \lambda\vec{y}, &/ \cdot P \text{ lewostronnie} \\
A(P\vec{y}) &= \lambda(P\vec{y}) \\
A\vec{x} &= \lambda\vec{x} \\
\vec{x} &= P\vec{y}
\end{alignat*}
Ostatnie równanie wyznacza przepis na wektory własne $ \vec{x}_k $ macierzy wejściowej. W zależności od wymiaru macierzy $ N $ uzyskuje się odpowiednio $ \vec{x}_k $ dla $ k = 1, 2, ..., N $. W celu sprawdzenia poprawności ich wyznaczenia stosowany jest współczynnik jakości obliczany dla każdego wektora jako

\begin{equation}
\beta_k = \frac{(x_k, Ax_k)}{(x_k, Ax_k)} \text{, gdzie}
\label{beta}
\end{equation}
\begin{itemize}
	\item$(x_k, Ax_k) $ - iloczyn skalarny macierzowy 
	\item$(x_k, x_k) $ - iloczyn skalarny danego wektora w przestrzeni euklidesowej.
\end{itemize}

\section{Zadanie do wykonania}

\subsection{Opis problemu}

Wzorem wstępu teoretycznego, należało przeprowadzić rozumowanie dla macierzy symetrycznej o wymiarze $ N = 5 $ danej przepisem $ A_{i,j} = \sqrt{i + j} \text{ dla } i,j \in \{1, 2, 3, 4, 5\} $. Jako że część wartości była konieczna do dalszego rozwiązywania zadania, szukano niewiadomych w podanej kolejności: 
\begin{itemize}
	\item macierz przekształcenia $ P $
	\item macierz trójdiagonalna $ T $
	\item wartości i wektory własne macierzy trójdiagonalnej $ T $, a na ich podstawie również dla macierzy wejściowej $ A $
	\item współczynniki jakości $ \beta_k $
\end{itemize}

\subsection{Wyniki}

Rozwiązanie opracowano w oparciu o bibliotekę\textit{ Numerical Recipes} ze szczególnym użyciem procedur \textit{tred2} oraz \textit{tqli}. Opracowana samodzielnie biblioteka \textit{ObjectiveNR} bazująca na \textit{NR} okazała się być bardzo pomocna przy operacjach macierzowo-wektorowych. Poniżej zaprezentowano otrzymane wyniki. \\

Macierz przekształcenia Householdera:
\begin{equation*}
P = 
\begin{pmatrix}
0.127995 &0.473318 &0.748056 &-0.447214 &0 \\ 
-0.559029 &-0.639797& 0.21169 &-0.483046 &0 \\
0.75337 &-0.341655 &-0.221449 &-0.516398 &0 \\
-0.321774& 0.499901& -0.588694& -0.547723& 0 \\
0& 0& 0& 0& 1 
\end{pmatrix}
\end{equation*}

Wartości własne macierzy A:
\begin{alignat*}{3}
\lambda_1 &= -2.4575e-07 \\
\lambda_2 &= -7.34517e-05 \\ 
\lambda_3 &= -0.00511678 \\
\lambda_4 &= -0.381893 \\
\lambda_5 &= 12.2415
\end{alignat*}

Wektory własne $\vec{y}_k \text{ macierzy } T \text{ dla odpowiednich wartości własnych } \lambda_k $:

\begin{equation*}
\vec{y}_1 = 
\begin{pmatrix}
-0.860161 \\ 0.371571 \\ -0.216227\\ -0.137207\\ -0.23765
\end{pmatrix} ,
\vec{y}_2 = 
\begin{pmatrix}
-0.509986 \\ -0.618499 \\ 0.369873\\ 0.234827\\ 0.406723
\end{pmatrix} ,
\vec{y}_3 = 
\begin{pmatrix}
-0.00604528 \\ -0.692329 \\ -0.436638\\-0.287573\\ -0.497285
\end{pmatrix} ,
\end{equation*}
\begin{equation*}
\vec{y}_4 = 
\begin{pmatrix}
1.00407e-06 \\ 0.00861619 \\ 0.790691\\-0.332559\\ -0.513943
\end{pmatrix},
\vec{y}_5 = 
\begin{pmatrix}
-2.99203e-11 \\8.22916e-06 \\-0.024372 \\-0.856 \\0.5164 
\end{pmatrix} 
\end{equation*}

Wektory własne $ \vec{x}_k $ macierzy $ A $ dla odpowiednich wartości własnych $ \lambda_k $:
\begin{equation*}
\vec{x}_1 = 
\begin{pmatrix}
-0.0346135  \\ 0.263629 \\ -0.656232\\0.664969 \\ -0.23765
\end{pmatrix},
\vec{x}_2 = 
\begin{pmatrix}
-0.186354  \\ 0.645677 \\ -0.376067 \\-0.49145  \\ 	0.406723
\end{pmatrix},
\vec{x}_3 = 
\begin{pmatrix}
-0.526489 \\0.492809  \\ 0.477178 \\0.0704053  \\ 	-0.497285
\end{pmatrix},
\end{equation*}
\begin{equation*}
\vec{x}_4 = 
\begin{pmatrix}
0.744284  \\0.32251  \\ -0.00630781 \\-0.279018  \\ -0.513943 
\end{pmatrix},
\vec{x}_5 = 
\begin{pmatrix}
0.364587 \\0.408323 \\0.447431  \\0.483203 \\0.5164 
\end{pmatrix}
\end{equation*}

\begin{table}[!hb]
	\centering
	\begin{tabular}{|c|c|c|}
		\hline
		k & $ \beta_k $& $ \lambda_k $  \\ \hline
		1&-5.60687e-07 & -2.4575e-07 \\ \hline
		2&-7.35992e-05 & -7.34517e-05 \\ \hline
		3&-0.00511705 & -0.00511678 \\ \hline
		4&-0.381893 & -0.381893 \\ \hline
		5&12.2415 & 12.2415 \\ \hline
	\end{tabular}
	\caption{Sprawdzenie, czy wektory $ \vec{x}_k $ są wektorami własnymi macierzy $ A $. Współczynniki postaci (\ref{beta}) powinny być równe wartościom własnym $ \lambda_k $.}
	\label{tabela}
\end{table}

\section{Wnioski}

Dzięki postaci macierzy wejściowej znajdowanie jej wektorów i wartości własnych okazało się dużo łatwiejsze i szybsze niż w typowych przypadkach. Wyznacznikiem poprawności wykonania zadania są współczynniki $ \beta_k $, które znikomo różnią się dla pierwszych trzech wartości, a dla czwartej i piątej są identyczne - zgodnie z tabelą (\ref{tabela}). Drobne rozbieżności mogą wynikać z reprezentacji liczb zmiennoprzecinkowych dla pojedynczej precyzji w pamięci komputera.

Całokształt wyników świadczy o dobrym sposobie opracowania rozwiązania i skuteczności opisywanej metody.

\end{document}